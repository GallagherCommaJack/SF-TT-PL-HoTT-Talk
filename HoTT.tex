\documentclass[xcolor=svgnames]{beamer}

\usepackage[utf8]    {inputenc}
\usepackage[T1]      {fontenc}
\usepackage[english] {babel}

\usepackage{amsmath,amsfonts,graphicx}
\usepackage{mathtools}
\usepackage{beamerleanprogress}

\title
  [Homotopy Type Theory\hspace{2em}]
  {Homotopy Type Theory}

%\subtitle{}

\author
  [Jack Gallagher]
  {Jack Gallagher}

\date{May 18, 2015}

% Operators
\DeclareMathOperator{\apf}{ap}
\include{macros}

\begin{document}

\maketitle

\section{Equality, Judgement, and Extensionality}
\begin{frame}{Judgemental Equality}
  \begin{itemize}
  \item Baked into evaluation
  \item ``Definitional''
  \item External to the logic
  \end{itemize}
\end{frame}

\begin{frame}{Propositional Equality}
  \begin{itemize}
  \item User defined
  \item Internal to the logic
  \item (In ITT) ``lifts'' judgemental equality
  \item Only one constructor: \pause
    $$ \mathrm{refl} : \prd{a:A} a =_A a $$
  \end{itemize} \pause

  What would an induction principle look like?
\end{frame}

\begin{frame}{Propositional Induction}
  Given: $A : U$, $C : \prd{a,b : A} a =_A b \to U$,
  and $c : \prd{a:A} C(a,a,\refl{a})$, we can define
  $$ f : \prd{a,b:A}{p : a =_A b} C(a,b,p) $$
  Such that for any $a:A$
  $$ f(a,a,\refl{a}) \defeq c(a) $$
\end{frame}

\begin{frame}{Problems}
  Propositional equality is weaker than we'd like
  \begin{itemize}
  \item It can't prove function extensionality \pause
  \item Lots of otherwise indistinguishable things are not equal \pause
  \item Quotients are hard \pause
  \end{itemize}

  How to solve this?
\end{frame}

\section{ETT}

\begin{frame}{Extensionality}
  ETT identifies judgemental and propositional equality \pause

  Upsides:
  \begin{itemize}
  \item Can prove more things
  \item Don't have to use transport
  \end{itemize} \pause

  Downsides:
  \begin{itemize}
  \item Proof checking is less well behaved
  \item Still doesn't solve the quotient problem
  \item Indistinguishable things still aren't equal
  \end{itemize}
\end{frame}

\section{Homotopy Type Theory}

\begin{frame}{Equalities Are Paths}
  % This is where coffee cup topology enters in
  The homotopical interpretation
  \begin{itemize}
  \item Types are spaces not sets
  \item Equalities are paths in that space
  \item Induction on equalities amounts to contracting paths
  \end{itemize}

  What do we get from this? % Spatial intuition is a powerful thing
\end{frame}

\begin{frame}{Transport}
  Like a degenerate version of path induction
  $$ \mathrm{transport} : \prd{P : A \to \type}{x,y : A} \to \id{x}{y} \to P(x) \to P(y) $$

  Can contract $p : \id{x}{y}$ to $\refl{x}$, so we have
  $\mathrm{transport}^P_{\refl{x}}(x,p) \defeq p$
\end{frame}

\begin{frame}{Functions and Continuity}
  We say that a function $f : A \to B$ is continuous if it preserves paths.
  $$ \apf_f : \prd{x,y:A} x =_A y \to f(x) =_B f(y) $$

  By path induction we can contract $p : \id{x}{y}$ to $\refl{x}$, for which we define
  $\apf_f(\refl{x}) \defeq \refl{f(x)}$ \pause

  Most of the important ideas in HoTT come from all functions being continuous
\end{frame}

\section{Univalence}

\begin{frame}
  \begin{center}
    \only<1>{\Huge When are two types equal?}
    \only<2>{\Huge When they represent equivalent spaces!}
  \end{center}
\end{frame}

\begin{frame}
   $$ \eqv{\id{A}{B}}{\eqv{A}{B}} $$

  What does this mean? \pause
   The real action is in the type $\eqv{A}{B}$

   What does it mean for two types to represent equivalent spaces? \pause
   
   $$
   \eqv{A}{B} \defeq \sm{f : A \to B}{g : B \to A}
   (g \circ f ~ \idfunc[A]) \times (f \circ g ~ \idfunc[B])
   $$
\end{frame}

\begin{frame}{Univalence and Transport}
  Univalence suggests we need a richer notion of transport. For any equivalence
  $f : A \to B$, the straightforward definition of transport is
  $$ transport^{\_ \mapsto A}_{\mathrm{ua}(f)}(A,a) = f(a) $$

  Equality proofs now carry information about \emph{how} to transport a value! \pause

  This is one of the most important notions from a PL perspective
  \begin{itemize}
  \item Predicates on isomorphic structures can be transported
  \item Lots of the benefits of Scala's implicits
  \item Libraries can share tools more easily
  \item Lets you give induction principles to nontrivial loop spaces
  \end{itemize}
\end{frame}

\section{Higher Inductive Types}

\begin{frame}
  \begin{center}
    \Huge Why not introduce other nontrivial equalities?
  \end{center}
\end{frame}

\begin{frame}{Beginning HIT's}
  The interval type:
  \begin{itemize}
  \item Two points $0_I,1_I : I$
  \item A path between them $\mathrm{seg} : \id{0_I}{1_I}$
  \end{itemize} \pause

  How do we keep this consistent? \pause By requiring continuity.
\end{frame}

\begin{frame}{Induction on HIT's}
  For any function $f : A \to B$, points $x,y : A$, and a path $p : x =_A y$,
  we need a value for $ap_f(p)$

  For $refl_x$ this is trivial, but what to do for $\map{f}{seg}$? \pause

  We include such a value in the definition of $f$!

  The induction principle for the interval:
  $$
  ind_I : \prd{P : I \to \type}{p_0 : P(0_I)}{p_1 : P(1_I)}
  p_0 =^P_{\mathrm{seg}} p_1 \to \prd{i : I} P(i)
  $$

  $$ ind_I(P,p_0,p_1,\_,0_I) \equiv p_0 $$
  $$ ind_I(P,p_0,p_1,\_,1_I) \equiv p_1 $$
  $$ \map{ind_I(P,\_,\_,p)}{seg} = p $$
\end{frame}

\begin{frame}{Function Extensionality}
  Extensionality is trivial with an interval \pause

  Given $f,g : A \to B$, and a proof $p : \prd{x:A} f(x) =_B g(x)$,
  we can define $q : A \times I \to B$
  \begin{align*}
    q(x, 0_I) &\defeq f(x) \\
    q(x, 1_I) &\defeq g(x) \\
    q(x, seg) &\defeq p(x)
  \end{align*} \pause
  $$ r(i) \defeq \lam{x} q(x, i) $$ \pause
  $$ \apf_r(seg) : f = g $$
\end{frame}

\begin{frame}{Quotient types}
  The reason HITs matter to non-topologists, show up everywhere

  Examples:
  \begin{itemize}
  \item Numbers (the rationals, the reals, an easy definition of integers)
  \item Languages (can make judgementally equal terms actually equal)
  \item Lots of other things (patch theory is a recent example)
  \end{itemize}
\end{frame}

\begin{frame}{Questions?}
  \begin{center}
    \Huge What else?
  \end{center}
\end{frame}
\end{document}
