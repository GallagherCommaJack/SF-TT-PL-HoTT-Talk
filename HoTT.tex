\documentclass[xcolor=svgnames]{beamer}

\usepackage{amsmath,amsfonts,amsthm,graphicx}
\usepackage{mathtools}
\usepackage{beamerleanprogress}

\input{macros}

\title
  [Homotopy Type Theory\hspace{2em}]
  {Homotopy Type Theory}

%\subtitle{}

\author
  [Jack Gallagher]
  {Jack Gallagher}

\date{May 18, 2015}

\begin{document}

\maketitle

\section{Equality, Judgement, and Extensionality}
\begin{frame}{Judgemental Equality}
  \begin{itemize}
  \item Baked into evaluation
  \item ``Definitional''
  \item External to the logic
  \end{itemize}
\end{frame}

\begin{frame}{Propositional Equality}
  \begin{itemize}
  \item User defined
  \item Internal to the logic
  \item (In ITT) ``lifts'' judgemental equality
  \item Only one constructor: \pause
    $$ \mathrm{refl} : \prd{a:A} a =_A a $$
  \end{itemize} \pause

  What would an induction principle look like?
\end{frame}

\begin{frame}{Propositional Induction}
  Given: $A : \type$, $C : \prd{a,b : A} a =_A b \to U$,
  and $c : \prd{a:A} C(a,a,\refl{a})$, we can define
  $$ f : \prd{a,b:A}{p : a =_A b} C(a,b,p) $$
  Such that for any $a:A$
  $$ f(a,a,\refl{a} ) \defeq c(a) $$
\end{frame}

\begin{frame}{Problems}
  Propositional equality is weaker than we'd like
  \begin{itemize}
  \item It can't prove function extensionality \pause
  \item Lots of otherwise indistinguishable things are not equal \pause
  \item Quotients are hard \pause
  \end{itemize}

  How to solve this?
\end{frame}

\section{ETT}

\begin{frame}{Extensionality}
  ETT identifies judgemental and propositional equality \pause

  Upsides:
  \begin{itemize}
  \item Can prove more things
  \item Don't have to use transport
  \end{itemize} \pause

  Downsides:
  \begin{itemize}
  \item Proof checking is less well behaved
  \item Still doesn't solve the quotient problem
  \item Indistinguishable things still aren't equal
  \end{itemize}
\end{frame}

\section{Homotopy Type Theory}

\begin{frame}{Equalities Are Paths}
  % This is where coffee cup topology enters in
  The homotopical interpretation
  \begin{itemize}
  \item Types are spaces not sets
  \item Equalities are paths in that space
  \item Induction on equalities amounts to contracting paths
  \end{itemize}

  What do we get from this? % Spatial intuition is a powerful thing
\end{frame}

\begin{frame}{Functions}
  We say that a function $f : A \to B$ is continuous if it preservs paths.
  $$ \apfunc{f} : \prd{x,y:A} x =_A y \to f(x) =_B f(y) $$

  By path induction we can contract $p : \id[A]{x}{y}$ to $\refl{x}$, for which we define
  $\apfunc{f}(\refl{x}) \defeq \refl{f(x)}$ \pause

  Most of the important ideas in HoTT come from all functions being continuous
\end{frame}

\section{Univalence}

\begin{frame}
  \begin{center}
    \only<1>{\Huge When are two types equal?}
    \only<2>{\Huge When they represent equivalent spaces!}
  \end{center}
\end{frame}

\begin{frame}
   $$ \eqv{\id[\type]{A}{B}}{\eqv{A}{B}} $$

   \only<1>{What does this mean?}
   \onslide<2->{The real action is in the type $\eqv{A}{B}$}

   \onslide<2->{What does it mean for two types to represent equivalent spaces?}
   \onslide<3->{
    $$
    \eqv{A}{B} \defeq \sm{f : A \to B}{g : B \to A}
    (g \circ f ~ \idfunc[A]) \times (f \circ g ~ \idfunc[B])
    $$}
\end{frame}

\section{Applications}

\subsection{Function Extensionality}
\subsection{Libraries}
\subsection{Quotients}
\subsection{Misc}

\section{Open problems}
\subsection{Enrichment}
\subsection{Metatheory}

\end{document}
